% Options for packages loaded elsewhere
\PassOptionsToPackage{unicode}{hyperref}
\PassOptionsToPackage{hyphens}{url}
%
\documentclass[
]{article}
\usepackage{amsmath,amssymb}
\usepackage{iftex}
\ifPDFTeX
  \usepackage[T1]{fontenc}
  \usepackage[utf8]{inputenc}
  \usepackage{textcomp} % provide euro and other symbols
\else % if luatex or xetex
  \usepackage{unicode-math} % this also loads fontspec
  \defaultfontfeatures{Scale=MatchLowercase}
  \defaultfontfeatures[\rmfamily]{Ligatures=TeX,Scale=1}
\fi
\usepackage{lmodern}
\ifPDFTeX\else
  % xetex/luatex font selection
\fi
% Use upquote if available, for straight quotes in verbatim environments
\IfFileExists{upquote.sty}{\usepackage{upquote}}{}
\IfFileExists{microtype.sty}{% use microtype if available
  \usepackage[]{microtype}
  \UseMicrotypeSet[protrusion]{basicmath} % disable protrusion for tt fonts
}{}
\makeatletter
\@ifundefined{KOMAClassName}{% if non-KOMA class
  \IfFileExists{parskip.sty}{%
    \usepackage{parskip}
  }{% else
    \setlength{\parindent}{0pt}
    \setlength{\parskip}{6pt plus 2pt minus 1pt}}
}{% if KOMA class
  \KOMAoptions{parskip=half}}
\makeatother
\usepackage{xcolor}
\usepackage[margin=1in]{geometry}
\usepackage{color}
\usepackage{fancyvrb}
\newcommand{\VerbBar}{|}
\newcommand{\VERB}{\Verb[commandchars=\\\{\}]}
\DefineVerbatimEnvironment{Highlighting}{Verbatim}{commandchars=\\\{\}}
% Add ',fontsize=\small' for more characters per line
\usepackage{framed}
\definecolor{shadecolor}{RGB}{248,248,248}
\newenvironment{Shaded}{\begin{snugshade}}{\end{snugshade}}
\newcommand{\AlertTok}[1]{\textcolor[rgb]{0.94,0.16,0.16}{#1}}
\newcommand{\AnnotationTok}[1]{\textcolor[rgb]{0.56,0.35,0.01}{\textbf{\textit{#1}}}}
\newcommand{\AttributeTok}[1]{\textcolor[rgb]{0.13,0.29,0.53}{#1}}
\newcommand{\BaseNTok}[1]{\textcolor[rgb]{0.00,0.00,0.81}{#1}}
\newcommand{\BuiltInTok}[1]{#1}
\newcommand{\CharTok}[1]{\textcolor[rgb]{0.31,0.60,0.02}{#1}}
\newcommand{\CommentTok}[1]{\textcolor[rgb]{0.56,0.35,0.01}{\textit{#1}}}
\newcommand{\CommentVarTok}[1]{\textcolor[rgb]{0.56,0.35,0.01}{\textbf{\textit{#1}}}}
\newcommand{\ConstantTok}[1]{\textcolor[rgb]{0.56,0.35,0.01}{#1}}
\newcommand{\ControlFlowTok}[1]{\textcolor[rgb]{0.13,0.29,0.53}{\textbf{#1}}}
\newcommand{\DataTypeTok}[1]{\textcolor[rgb]{0.13,0.29,0.53}{#1}}
\newcommand{\DecValTok}[1]{\textcolor[rgb]{0.00,0.00,0.81}{#1}}
\newcommand{\DocumentationTok}[1]{\textcolor[rgb]{0.56,0.35,0.01}{\textbf{\textit{#1}}}}
\newcommand{\ErrorTok}[1]{\textcolor[rgb]{0.64,0.00,0.00}{\textbf{#1}}}
\newcommand{\ExtensionTok}[1]{#1}
\newcommand{\FloatTok}[1]{\textcolor[rgb]{0.00,0.00,0.81}{#1}}
\newcommand{\FunctionTok}[1]{\textcolor[rgb]{0.13,0.29,0.53}{\textbf{#1}}}
\newcommand{\ImportTok}[1]{#1}
\newcommand{\InformationTok}[1]{\textcolor[rgb]{0.56,0.35,0.01}{\textbf{\textit{#1}}}}
\newcommand{\KeywordTok}[1]{\textcolor[rgb]{0.13,0.29,0.53}{\textbf{#1}}}
\newcommand{\NormalTok}[1]{#1}
\newcommand{\OperatorTok}[1]{\textcolor[rgb]{0.81,0.36,0.00}{\textbf{#1}}}
\newcommand{\OtherTok}[1]{\textcolor[rgb]{0.56,0.35,0.01}{#1}}
\newcommand{\PreprocessorTok}[1]{\textcolor[rgb]{0.56,0.35,0.01}{\textit{#1}}}
\newcommand{\RegionMarkerTok}[1]{#1}
\newcommand{\SpecialCharTok}[1]{\textcolor[rgb]{0.81,0.36,0.00}{\textbf{#1}}}
\newcommand{\SpecialStringTok}[1]{\textcolor[rgb]{0.31,0.60,0.02}{#1}}
\newcommand{\StringTok}[1]{\textcolor[rgb]{0.31,0.60,0.02}{#1}}
\newcommand{\VariableTok}[1]{\textcolor[rgb]{0.00,0.00,0.00}{#1}}
\newcommand{\VerbatimStringTok}[1]{\textcolor[rgb]{0.31,0.60,0.02}{#1}}
\newcommand{\WarningTok}[1]{\textcolor[rgb]{0.56,0.35,0.01}{\textbf{\textit{#1}}}}
\usepackage{graphicx}
\makeatletter
\def\maxwidth{\ifdim\Gin@nat@width>\linewidth\linewidth\else\Gin@nat@width\fi}
\def\maxheight{\ifdim\Gin@nat@height>\textheight\textheight\else\Gin@nat@height\fi}
\makeatother
% Scale images if necessary, so that they will not overflow the page
% margins by default, and it is still possible to overwrite the defaults
% using explicit options in \includegraphics[width, height, ...]{}
\setkeys{Gin}{width=\maxwidth,height=\maxheight,keepaspectratio}
% Set default figure placement to htbp
\makeatletter
\def\fps@figure{htbp}
\makeatother
\setlength{\emergencystretch}{3em} % prevent overfull lines
\providecommand{\tightlist}{%
  \setlength{\itemsep}{0pt}\setlength{\parskip}{0pt}}
\setcounter{secnumdepth}{-\maxdimen} % remove section numbering
\ifLuaTeX
  \usepackage{selnolig}  % disable illegal ligatures
\fi
\usepackage{bookmark}
\IfFileExists{xurl.sty}{\usepackage{xurl}}{} % add URL line breaks if available
\urlstyle{same}
\hypersetup{
  pdftitle={hw1},
  pdfauthor={Korotkov Vitaliy},
  hidelinks,
  pdfcreator={LaTeX via pandoc}}

\title{hw1}
\author{Korotkov Vitaliy}
\date{2024-10-19}

\begin{document}
\maketitle

\subsection{Работа с
данными}\label{ux440ux430ux431ux43eux442ux430-ux441-ux434ux430ux43dux43dux44bux43cux438}

\subsubsection{Загрузка
данных}\label{ux437ux430ux433ux440ux443ux437ux43aux430-ux434ux430ux43dux43dux44bux445}

\begin{Shaded}
\begin{Highlighting}[]
\NormalTok{data.df }\OtherTok{\textless{}{-}} \FunctionTok{read.table}\NormalTok{(}\StringTok{"http://people.math.umass.edu/\textasciitilde{}anna/Stat597AFall2016/rnf6080.dat"}\NormalTok{, }\AttributeTok{header=}\ConstantTok{FALSE}\NormalTok{)}
\FunctionTok{cat}\NormalTok{(}\StringTok{"Количество строк: "}\NormalTok{, }\FunctionTok{nrow}\NormalTok{(data.df), }\StringTok{"}\SpecialCharTok{\textbackslash{}n}\StringTok{Количество столбцов: "}\NormalTok{, }\FunctionTok{ncol}\NormalTok{(data.df))}
\end{Highlighting}
\end{Shaded}

\begin{verbatim}
## Количество строк:  5070 
## Количество столбцов:  27
\end{verbatim}

\subsubsection{Имена
колонок}\label{ux438ux43cux435ux43dux430-ux43aux43eux43bux43eux43dux43eux43a}

\begin{Shaded}
\begin{Highlighting}[]
\FunctionTok{colnames}\NormalTok{(data.df)}
\end{Highlighting}
\end{Shaded}

\begin{verbatim}
##  [1] "V1"  "V2"  "V3"  "V4"  "V5"  "V6"  "V7"  "V8"  "V9"  "V10" "V11" "V12"
## [13] "V13" "V14" "V15" "V16" "V17" "V18" "V19" "V20" "V21" "V22" "V23" "V24"
## [25] "V25" "V26" "V27"
\end{verbatim}

\subsubsection{Значение 5 строки 7
столбца}\label{ux437ux43dux430ux447ux435ux43dux438ux435-5-ux441ux442ux440ux43eux43aux438-7-ux441ux442ux43eux43bux431ux446ux430}

\begin{Shaded}
\begin{Highlighting}[]
\NormalTok{data.df[}\DecValTok{5}\NormalTok{, }\DecValTok{7}\NormalTok{]}
\end{Highlighting}
\end{Shaded}

\begin{verbatim}
## [1] 0
\end{verbatim}

\subsubsection{Вторая
строка}\label{ux432ux442ux43eux440ux430ux44f-ux441ux442ux440ux43eux43aux430}

\begin{Shaded}
\begin{Highlighting}[]
\NormalTok{data.df[}\DecValTok{2}\NormalTok{, ]}
\end{Highlighting}
\end{Shaded}

\begin{verbatim}
##   V1 V2 V3 V4 V5 V6 V7 V8 V9 V10 V11 V12 V13 V14 V15 V16 V17 V18 V19 V20 V21
## 2 60  4  2  0  0  0  0  0  0   0   0   0   0   0   0   0   0   0   0   0   0
##   V22 V23 V24 V25 V26 V27
## 2   0   0   0   0   0   0
\end{verbatim}

\subsubsection{Замена заголовков
столбцов}\label{ux437ux430ux43cux435ux43dux430-ux437ux430ux433ux43eux43bux43eux432ux43aux43eux432-ux441ux442ux43eux43bux431ux446ux43eux432}

\begin{Shaded}
\begin{Highlighting}[]
\FunctionTok{names}\NormalTok{(data.df) }\OtherTok{\textless{}{-}} \FunctionTok{c}\NormalTok{(}\StringTok{"year"}\NormalTok{, }\StringTok{"month"}\NormalTok{, }\StringTok{"day"}\NormalTok{, }\FunctionTok{seq}\NormalTok{(}\DecValTok{0}\NormalTok{, }\DecValTok{23}\NormalTok{))}
\end{Highlighting}
\end{Shaded}

Данная строка заменяет заголовки столбцов на ``year'', ``month'',
``day'', и 0, 1, 2, \ldots, 23, которые соответствуют часу дня, в
который были зафиксированы осадки.

Начало таблицы:

\begin{Shaded}
\begin{Highlighting}[]
\FunctionTok{head}\NormalTok{(data.df)}
\end{Highlighting}
\end{Shaded}

\begin{verbatim}
##   year month day 0 1 2 3 4 5 6 7 8 9 10 11 12 13 14 15 16 17 18 19 20 21 22 23
## 1   60     4   1 0 0 0 0 0 0 0 0 0 0  0  0  0  0  0  0  0  0  0  0  0  0  0  0
## 2   60     4   2 0 0 0 0 0 0 0 0 0 0  0  0  0  0  0  0  0  0  0  0  0  0  0  0
## 3   60     4   3 0 0 0 0 0 0 0 0 0 0  0  0  0  0  0  0  0  0  0  0  0  0  0  0
## 4   60     4   4 0 0 0 0 0 0 0 0 0 0  0  0  0  0  0  0  0  0  0  0  0  0  0  0
## 5   60     4   5 0 0 0 0 0 0 0 0 0 0  0  0  0  0  0  0  0  0  0  0  0  0  0  0
## 6   60     4   6 0 0 0 0 0 0 0 0 0 0  0  0  0  0  0  0  0  0  0  0  0  0  0  0
\end{verbatim}

Конец таблицы:

\begin{Shaded}
\begin{Highlighting}[]
\FunctionTok{tail}\NormalTok{(data.df)}
\end{Highlighting}
\end{Shaded}

\begin{verbatim}
##      year month day 0 1 2 3 4 5 6 7 8 9 10 11 12 13 14 15 16 17 18 19 20 21 22
## 5065   80    11  25 0 0 0 0 0 0 0 0 0 0  0  0  0  0  0  0  0  0  0  0  0  0  0
## 5066   80    11  26 0 0 0 0 0 0 0 0 0 0  0  0  0  0  0  0  0  0  0  0  0  0  0
## 5067   80    11  27 0 0 0 0 0 0 0 0 0 0  0  0  0  0  0  0  0  0  0  0  0  0  0
## 5068   80    11  28 0 0 0 0 0 0 0 0 0 0  0  0  0  0  0  0  0  0  0  0  0  0  0
## 5069   80    11  29 0 0 0 0 0 0 0 0 0 0  0  0  0  0  0  0  0  0  0  0  0  0  0
## 5070   80    11  30 0 0 0 0 0 0 0 0 0 0  0  0  0  0  0  0  0  0  0  0  0  0  0
##      23
## 5065  0
## 5066  0
## 5067  0
## 5068  0
## 5069  0
## 5070  0
\end{verbatim}

Последние 24 колонки представляют собой количество осадков по каждому
часу дня.

\subsubsection{Добавление колонки и построение
гистограммы}\label{ux434ux43eux431ux430ux432ux43bux435ux43dux438ux435-ux43aux43eux43bux43eux43dux43aux438-ux438-ux43fux43eux441ux442ux440ux43eux435ux43dux438ux435-ux433ux438ux441ux442ux43eux433ux440ux430ux43cux43cux44b}

Добавим новую колонку \texttt{daily}, которая будет содержать сумму
осадков за день (по всем часам):

\begin{Shaded}
\begin{Highlighting}[]
\NormalTok{data.df}\SpecialCharTok{$}\NormalTok{daily }\OtherTok{\textless{}{-}} \FunctionTok{rowSums}\NormalTok{(data.df[, }\DecValTok{4}\SpecialCharTok{:}\DecValTok{27}\NormalTok{])}
\FunctionTok{head}\NormalTok{(data.df)}
\end{Highlighting}
\end{Shaded}

\begin{verbatim}
##   year month day 0 1 2 3 4 5 6 7 8 9 10 11 12 13 14 15 16 17 18 19 20 21 22 23
## 1   60     4   1 0 0 0 0 0 0 0 0 0 0  0  0  0  0  0  0  0  0  0  0  0  0  0  0
## 2   60     4   2 0 0 0 0 0 0 0 0 0 0  0  0  0  0  0  0  0  0  0  0  0  0  0  0
## 3   60     4   3 0 0 0 0 0 0 0 0 0 0  0  0  0  0  0  0  0  0  0  0  0  0  0  0
## 4   60     4   4 0 0 0 0 0 0 0 0 0 0  0  0  0  0  0  0  0  0  0  0  0  0  0  0
## 5   60     4   5 0 0 0 0 0 0 0 0 0 0  0  0  0  0  0  0  0  0  0  0  0  0  0  0
## 6   60     4   6 0 0 0 0 0 0 0 0 0 0  0  0  0  0  0  0  0  0  0  0  0  0  0  0
##   daily
## 1     0
## 2     0
## 3     0
## 4     0
## 5     0
## 6     0
\end{verbatim}

Построим гистограмму для новой колонки \texttt{daily}:

\begin{Shaded}
\begin{Highlighting}[]
\FunctionTok{hist}\NormalTok{(data.df}\SpecialCharTok{$}\NormalTok{daily, }\AttributeTok{main =} \StringTok{"Количество осадков по дням"}\NormalTok{, }\AttributeTok{xlab =} \StringTok{"Осадки"}\NormalTok{, }\AttributeTok{ylab =} \StringTok{"Количество дней"}\NormalTok{)}
\end{Highlighting}
\end{Shaded}

\includegraphics{hw1_files/figure-latex/unnamed-chunk-9-1.pdf}

Мы видим, что в данных есть некорректные значения (-999), что приводит к
неправильной интерпретации гистограммы.

\subsubsection{Исправление ошибок в данных и создание нового
датафрейма}\label{ux438ux441ux43fux440ux430ux432ux43bux435ux43dux438ux435-ux43eux448ux438ux431ux43eux43a-ux432-ux434ux430ux43dux43dux44bux445-ux438-ux441ux43eux437ux434ux430ux43dux438ux435-ux43dux43eux432ux43eux433ux43e-ux434ux430ux442ux430ux444ux440ux435ux439ux43cux430}

Удалим некорректные значения (-999):

\begin{Shaded}
\begin{Highlighting}[]
\NormalTok{fixed.df }\OtherTok{\textless{}{-}}\NormalTok{ data.df[data.df}\SpecialCharTok{$}\NormalTok{daily }\SpecialCharTok{\textgreater{}} \DecValTok{0}\NormalTok{, ]}
\end{Highlighting}
\end{Shaded}

Построим гистограмму для исправленного датафрейма:

\begin{Shaded}
\begin{Highlighting}[]
\FunctionTok{hist}\NormalTok{(fixed.df}\SpecialCharTok{$}\NormalTok{daily, }\AttributeTok{main =} \StringTok{"Количество осадков по дням (исправленные данные)"}\NormalTok{, }\AttributeTok{xlab =} \StringTok{"Осадки"}\NormalTok{, }\AttributeTok{ylab =} \StringTok{"Количество дней"}\NormalTok{, }\AttributeTok{breaks =} \DecValTok{80}\NormalTok{)}
\end{Highlighting}
\end{Shaded}

\includegraphics{hw1_files/figure-latex/unnamed-chunk-11-1.pdf}

Новая гистограмма более корректна, так как она построена на данных без
ошибочных значений.

\subsection{Синтаксис и
типизирование}\label{ux441ux438ux43dux442ux430ux43aux441ux438ux441-ux438-ux442ux438ux43fux438ux437ux438ux440ux43eux432ux430ux43dux438ux435}

\subsubsection{Пример 1}\label{ux43fux440ux438ux43cux435ux440-1}

\begin{Shaded}
\begin{Highlighting}[]
\NormalTok{v }\OtherTok{\textless{}{-}} \FunctionTok{c}\NormalTok{(}\StringTok{"4"}\NormalTok{, }\StringTok{"8"}\NormalTok{, }\StringTok{"15"}\NormalTok{, }\StringTok{"16"}\NormalTok{, }\StringTok{"23"}\NormalTok{, }\StringTok{"42"}\NormalTok{)}
\FunctionTok{max}\NormalTok{(v)}
\end{Highlighting}
\end{Shaded}

\begin{verbatim}
## [1] "8"
\end{verbatim}

Функция \texttt{max()} ищет максимальное значение вектора. Поскольку все
элементы вектора являются строками, они сравниваются лексикографически,
по первому символу.

\begin{Shaded}
\begin{Highlighting}[]
\FunctionTok{sort}\NormalTok{(v)}
\end{Highlighting}
\end{Shaded}

\begin{verbatim}
## [1] "15" "16" "23" "4"  "42" "8"
\end{verbatim}

Функция \texttt{sort()} сортирует строки в алфавитном порядке, опять же
лексикографически.

\begin{Shaded}
\begin{Highlighting}[]
\CommentTok{\# sum(v)}
\end{Highlighting}
\end{Shaded}

Функция \texttt{sum()} не работает для строковых векторов, выдаст
ошибку.

\subsubsection{Пример 2}\label{ux43fux440ux438ux43cux435ux440-2}

\begin{Shaded}
\begin{Highlighting}[]
\CommentTok{\# v2 \textless{}{-} c("5", 7, 12)}
\CommentTok{\# v2[2] + v2[3]}
\end{Highlighting}
\end{Shaded}

Хотя элементы вектора представлены как числа, вектор смешанного типа
автоматически преобразуется в строковый. Следовательно, сложение вызовет
ошибку.

\begin{Shaded}
\begin{Highlighting}[]
\NormalTok{df3 }\OtherTok{\textless{}{-}} \FunctionTok{data.frame}\NormalTok{(}\AttributeTok{z1=}\StringTok{"5"}\NormalTok{, }\AttributeTok{z2=}\DecValTok{7}\NormalTok{, }\AttributeTok{z3=}\DecValTok{12}\NormalTok{)}
\NormalTok{df3[}\DecValTok{1}\NormalTok{, }\DecValTok{2}\NormalTok{] }\SpecialCharTok{+}\NormalTok{ df3[}\DecValTok{1}\NormalTok{, }\DecValTok{3}\NormalTok{]}
\end{Highlighting}
\end{Shaded}

\begin{verbatim}
## [1] 19
\end{verbatim}

В данном примере элементы второго и третьего столбца могут быть сложены,
так как они являются числовыми.

\begin{Shaded}
\begin{Highlighting}[]
\NormalTok{l4 }\OtherTok{\textless{}{-}} \FunctionTok{list}\NormalTok{(}\AttributeTok{z1=}\StringTok{"6"}\NormalTok{, }\AttributeTok{z2=}\DecValTok{42}\NormalTok{, }\AttributeTok{z3=}\StringTok{"49"}\NormalTok{, }\AttributeTok{z4=}\DecValTok{126}\NormalTok{)}
\NormalTok{l4[[}\DecValTok{2}\NormalTok{]] }\SpecialCharTok{+}\NormalTok{ l4[[}\DecValTok{4}\NormalTok{]]}
\end{Highlighting}
\end{Shaded}

\begin{verbatim}
## [1] 168
\end{verbatim}

Сложение значений из второго и четвертого элементов списка работает, так
как оба являются числами.

\begin{Shaded}
\begin{Highlighting}[]
\CommentTok{\# l4[2] + l4[4]}
\end{Highlighting}
\end{Shaded}

Это вызовет ошибку, так как обращение \texttt{l4{[}2{]}} и
\texttt{l4{[}4{]}} возвращает список, а не числовые значения.

\subsection{Работа с функциями и
операторами}\label{ux440ux430ux431ux43eux442ux430-ux441-ux444ux443ux43dux43aux446ux438ux44fux43cux438-ux438-ux43eux43fux435ux440ux430ux442ux43eux440ux430ux43cux438}

\subsubsection{Последовательности
чисел}\label{ux43fux43eux441ux43bux435ux434ux43eux432ux430ux442ux435ux43bux44cux43dux43eux441ux442ux438-ux447ux438ux441ux435ux43b}

Числа от 1 до 10000 с инкрементом 372:

\begin{Shaded}
\begin{Highlighting}[]
\FunctionTok{seq}\NormalTok{(}\DecValTok{1}\NormalTok{, }\DecValTok{10000}\NormalTok{, }\AttributeTok{by=}\DecValTok{372}\NormalTok{)}
\end{Highlighting}
\end{Shaded}

\begin{verbatim}
##  [1]    1  373  745 1117 1489 1861 2233 2605 2977 3349 3721 4093 4465 4837 5209
## [16] 5581 5953 6325 6697 7069 7441 7813 8185 8557 8929 9301 9673
\end{verbatim}

Числа от 1 до 10000 длиной 50:

\begin{Shaded}
\begin{Highlighting}[]
\FunctionTok{seq}\NormalTok{(}\DecValTok{1}\NormalTok{, }\DecValTok{10000}\NormalTok{, }\AttributeTok{length.out=}\DecValTok{50}\NormalTok{)}
\end{Highlighting}
\end{Shaded}

\begin{verbatim}
##  [1]     1.0000   205.0612   409.1224   613.1837   817.2449  1021.3061
##  [7]  1225.3673  1429.4286  1633.4898  1837.5510  2041.6122  2245.6735
## [13]  2449.7347  2653.7959  2857.8571  3061.9184  3265.9796  3470.0408
## [19]  3674.1020  3878.1633  4082.2245  4286.2857  4490.3469  4694.4082
## [25]  4898.4694  5102.5306  5306.5918  5510.6531  5714.7143  5918.7755
## [31]  6122.8367  6326.8980  6530.9592  6735.0204  6939.0816  7143.1429
## [37]  7347.2041  7551.2653  7755.3265  7959.3878  8163.4490  8367.5102
## [43]  8571.5714  8775.6327  8979.6939  9183.7551  9387.8163  9591.8776
## [49]  9795.9388 10000.0000
\end{verbatim}

\subsubsection{\texorpdfstring{Функция
\texttt{rep()}}{Функция rep()}}\label{ux444ux443ux43dux43aux446ux438ux44f-rep}

\begin{Shaded}
\begin{Highlighting}[]
\FunctionTok{rep}\NormalTok{(}\DecValTok{1}\SpecialCharTok{:}\DecValTok{5}\NormalTok{, }\AttributeTok{times=}\DecValTok{3}\NormalTok{)}
\end{Highlighting}
\end{Shaded}

\begin{verbatim}
##  [1] 1 2 3 4 5 1 2 3 4 5 1 2 3 4 5
\end{verbatim}

Этот код повторяет весь вектор \texttt{1:5} три раза.

\begin{Shaded}
\begin{Highlighting}[]
\FunctionTok{rep}\NormalTok{(}\DecValTok{1}\SpecialCharTok{:}\DecValTok{5}\NormalTok{, }\AttributeTok{each=}\DecValTok{3}\NormalTok{)}
\end{Highlighting}
\end{Shaded}

\begin{verbatim}
##  [1] 1 1 1 2 2 2 3 3 3 4 4 4 5 5 5
\end{verbatim}

Этот код повторяет каждый элемент вектора три раза.

\end{document}
